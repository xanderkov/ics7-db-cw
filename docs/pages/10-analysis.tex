\section{Аналитическая часть}

В данном разделе проведен анализ предметной области, формализированы данные,
а также проведен анализ существующих решений.

\subsection{Анализ предметной области}

Термин "база данных" не имеет точного определения, но стоит отметить несколько из них.


\textbf{База данных} \cite{bd-1} --- это совокупность данных, хранимых в упорядоченной форме, с целью
обеспечения доступа к этим данным и их использования каким-либо организационными
или прикладными процессам.

\textbf{База данных} --- это самодокументирования собрание интегрированных записей.

\begin{itemize}
    \item Запись --- это события, которые надо где-то хранить;
    \item интегрированных --- записи, которые имеют некоторую структуру.
\end{itemize}

Также необходимо определиться с типом базы данных. 
Всего существует два основных применения баз данных

\begin{enumerate}[label=\arabic*)]
    \item OLAP --- это метод обработки данных, который используется для анализа
                    больших объемов данных.
    \item OLTP --- это метод обаработки транзакций, который используется для
                    выполнения операция в режиме реального времени.
\end{enumerate}

Из этих данных определений следует, что для поставленной задачи больше подойдет метод
OLTP, так как для обработки посетителей в магазине, требуется обработка в реальном времени.

Для выполнения курсовой работы, также необходимо выбрать систему управления базами данных.

\textbf{СУБД} --- это приложение обеспечивающее создание, хранение, обновление и поиск информации.

У систем управления базами данных существует классификация: 

\subsubsection{Классификация СУБД}
\begin{enumerate}
	\item По модели данных:
	\begin{itemize}
		\item Дореляционные (Инвертированные списки, иерархические и сетевые)
		\begin{itemize}
			\item Инвертированные списки (файлы). БД на основе инвертированных списков представляет собой совокупность файлов, содержащих записи (таблиц). Для записей в файле определен некоторый порядок, диктуемый физической организацией данных. Для каждого файла может быть определено произвольное число других упорядочений на основании значений некоторых полей записей (инвертированных списков). Обычно для этого используются индексы. В такой модели данных отсутствуют ограничения целостности как таковые. Все ограничения на возможные экземпляры БД задаются теми программами, которые работают с БД. Одно из немногих ограничений, которое все-таки может присутствовать - это ограничение, задаваемое уникальным индексом. 
			\item Иерархичекие
			\item Сетевые (могут быть представлены в виде графа; логика выборки зависит от физической организации данных)
		\end{itemize}
		\item Реляционные
		\begin{itemize}
			\item Структурный (данные --- набор отношений)
			\item Целостностный (отношения (таблицы) отвечают определенным условиям целостности)
			\item Манипуляционный (манипулирования отношениями осуществляется средствами реляционной алгебры и/или реляционного исчисления)
		\end{itemize}
		\item Постреляционные
	\end{itemize}
	\item По архитектуре организации хранения данных:
	\begin{itemize}
		\item Локальные (все части локальной СУБД размещаются на одном компьютере)
		\item Распределенные (части СУБД могут размещаться на 2-х и более компьютерах) 
	\end{itemize}
	\item По способу доступа к БД:
	\begin{itemize}
		\item Файл-серверные (при работе с базой, данные перегоняются приложению, которое с ней работает, вне зависимости от того, сколько их нужно. Все операции --- на стороне клиента. Файловый сервер периодически обновляется тем же клиентом)
		\item Клиент-серверные (вся работа на сервере, по сети передаются результаты запросов, гораздо меньше информации. Обеспечивается безопасность данных, потому что все происходит на стороне сервера. Проще исключить одновременное изменение и тп)
		\item Встраиваемые --- библиотека, которая позволяет унифицированным образом хранить 
		большие объемы данных на локальной машине. Доступ к данным может происходить через SQL либо через 
		особые функции СУБД. Встраиваемые СУБД быстрее обычных клиент-серверных и не требуют установки 
		сервера, поэтому востребованы в локальном ПО, которое имеет дело с большими объемами данных.
		\item Сервисно-ориентированные  (БД является хранилищем сообщений, промежуточных состояний, метаинформации об очередях сообщений и сервисах)
		\item Прочие (пространственная, временная и пространственно-временная)
	\end{itemize}
\end{enumerate}

\subsection{Формализация данных}


\subsection*{Вывод}
