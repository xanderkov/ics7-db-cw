\section{Технологическая часть}

В данном разделе выбирается СУБД, средства реализации приложения,
описаны создание базы данных, триггера, функции и ролей,
а также спроектирован пользовательский интерфейс.

\subsection{Выбор СУБД}

Существует множество различных СУБД, работающих на основе реляционной модели,
каждая из которых имеет свои сильные и слабые стороны.
Среди самых распространенных \cite{most_popular} выделяют MySQL \cite{mysql},
PostgreSQL \cite{postgresql} и SQLite \cite{sqlite}.
Рассмотрим особенности каждой из них.

\begin{enumerate}
    \item MySQL. Среди достоинств данной СУБД можно выделить высокую
          безопасность и масштабируемость,
          поддержку большей части функционала SQL. Однако, несмотря на перечисленные
          положительные аспекты,
          MySQL не сопровождается бесплатной технической поддержкой.
    \item PostgreSQL. В рамках использования этой СУБД имеется возможность
          помимо встроенного SQL использовать различные дополнения,
          отличается поддержкой форматов csv и json, но оперирует большим объемом
          ресурсов.
    \item SQLite. Очевидными достоинствами является компактность базы
          данных, которая состоит из одного файла,
          и переносимость.
          Но данная СУБД совершенно не подходит для больших БД, а также не
          поддерживает управление пользователями.
\end{enumerate}

При реализации проекта использован PostgreSQL,
поскольку эта СУБД обладает достаточным набором инструментов для поставленной
задачи.

\subsection{Создание базы данных}

В качестве используемого языка программирования выбран Python \cite{python}, так как он удовлетворяет всем требованиям.